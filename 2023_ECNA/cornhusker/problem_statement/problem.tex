\problemname{Cornhusker}

Corn farmers need to do pre-harvest yield estimates to determine the
approximate number of bushels of corn their farm will produce. They do
this to determine if they have enough storage space (grain bins) to
store the harvested crop or if they'll have to store the corn elsewhere,
like a co-op (which costs \$\$\$). They also use these estimates when
negotiating the future market prices of their corn. Estimates are
typically done about a month or two before harvest.  By this time, the ears have
formed and the kernels on the ears are mostly developed (this makes
counting the kernels easier).

According to the \emph{University of Nebraska-Lincoln}, \emph{Nebraska
Extension for Educators}, the standard way to estimate corn yield is to
calculate the number of bushels of corn per acre. To make the
calculations easier, they use an area of 1/1000th of an acre, which,
with 30'' row spacing, is a section of one row about 17'5'' long. Within
that 17'5'', five ears are chosen at random. For each ear, the number of
kernels are counted by multiplying the number of rows of kernels around
by the number of kernels over the length of the ear. The totals for each
of the five ears are added together and then divided by five to
determine the average number of kernels per ear of corn. This number is
then multiplied by the total number of ears of corn in the 17'5''
section of row. This gives you the total number of kernels in 1/1000th
of an acre. This number is then divided by the \emph{Kernel Weight
Factor} (\emph{KWF}). The \emph{KWF} is a function of how wet (or dry)
the growing season is and is typically a value between 75 (wet) and 95
(dry). The resulting quotient is the number of bushels/acre the farmer
can expect to harvest.

For example, suppose that the average number of kernels per ear is 512
(16 kernels around by 32 kernels lengthwise), and there are 25 ears in
the 17'5'' of row with a \emph{KWF} of 85. The farmer could then expect:

\begin{center}
{\LARGE \(\frac{25\times 512}{85}\  = \ 150\ bushels\)}
\end{center}

Since farmers are quite conservative in their estimates, all
calculations are done as integers with no rounding.

\section*{Input}
Input consists of two lines. The first line contains 10 space separated
integer values representing the number of kernels around (\emph{A}) and
number of kernels long (\emph{L}) for each of five ears of corn
($8 \leq \emph{A} \leq 24$), ($20 \leq \emph{L} \leq 50$).

The second line contains 2 space
separated integer values representing the number of ears of corn,
\emph{N}, in the 17'5'' row ($10 \leq \emph{N} \leq 50$) and the \emph{KWF}
($75 \leq \emph{KWF} \leq 95$).

\section*{Output}

Output a single integer equal to the estimated number of bushels of corn
per acre the farmer can expect given the input supplied.
\pagebreak
